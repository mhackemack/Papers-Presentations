\documentclass{anstrans}
%%%%%%%%%%%%%%%%%%%%%%%%%%%%%%%%%%%
\title{A DFEM Formulation of the Diffusion Equation on Arbitrary Polyhedral Grids}
\author{Michael W. Hackemack,$^{*}$ Jean C. Ragusa$^{*}$}

\institute{
$^{*}$Department of Nuclear Engineering, Texas A\&M Univeristy, 
}

\email{mike\_hack@tamu.edu \and ragusa@ne.tamu.edu}

% Optional disclaimer: remove this command to hide
%\disclaimer{Notice: This manuscript is a work of fiction. Any resemblance to actual articles, living or dead, is purely coincidental.}

%%%% packages and definitions (optional)
\usepackage{graphicx} % allows inclusion of graphics
\usepackage{booktabs} % nice rules (thick lines) for tables
\usepackage{microtype} % improves typography for PDF

\newcommand{\SN}{S$_N$}
\renewcommand{\vec}[1]{\bm{#1}} %vector is bold italic
\newcommand{\vd}{\bm{\cdot}} % slightly bold vector dot
\newcommand{\grad}{\vec{\nabla}} % gradient
\newcommand{\ud}{\mathop{}\!\mathrm{d}} % upright derivative symbol

\begin{document}
%%%%%%%%%%%%%%%%%%%%%%%%%%%%%%%%%%%%%%%%%%%%%%%%%%%%%%%%%%%%%%%%%%%%%%%%%%%%%%%%
\section{Introduction}


%%%%%%%%%%%%%%%%%%%%%%%%%%%%%%%%%%%%%%%%%%%%%%%%%%%%%%%%%%%%%%%%%%%%%%%%%%%%%%%%
\section{Theory}

For this work, we will analyze a simple form of the diffusion equation that can be used for either monoenergetic neutron diffusion or the grey diffusion of radiation transport \cite{duderstadt1976nuclear}. This diffusion equation for this simple system is given by the following:

\begin{equation} \label{eq:DIffEq}
- \vec{\nabla} \cdot D \vec{\nabla} \Phi(\vec{x}) + \sigma \Phi(\vec{x}) = q(\vec{x}), \qquad x \in \mathcal{D}
\end{equation}

\noindent where $D$ is the diffusion coefficient, $\sigma$ is the total cross section, $q(\vec{x})$ is some source that can consist of fission and scattering terms as well as some distributed source, and $\Phi(\vec{x})$ is the scalar flux solution. The boundary conditions for this diffusion problem are

\begin{equation} \label{eq:dirichlet_bndy}
\Phi \, =\Phi_0 \qquad \vec{x} \in \partial \mathcal{D}^d
\end{equation}

\noindent for the Dirichlet boundaries,

\begin{equation} \label{eq:neumann_bndy}
- D \partial_n \Phi \, = J_{0} \qquad \vec{x} \in \partial \mathcal{D}^n
\end{equation}

\noindent for the Neumann boundaries, and

\begin{equation} \label{eq:robin_bndy}
 \frac{1}{4} \Phi + \frac{1}{2} D \partial_n \Phi \, = J^{inc} \qquad \vec{x} \in \partial \mathcal{D}^r
\end{equation}

\noindent for the Robin boundaries where the whole domain boundary: $\partial \mathcal{D} \in \mathcal{D}^d \cup \mathcal{D}^n \cup \mathcal{D}^r$. $\partial_n = \vec{n} \cdot \vec{\nabla}$ is defined as the partial derivative of the outward normal to the surface.

Now that we have described the physical problem to be solved, we next define 

%%%%%%%%%%%%%%%%%%%%%%%%%%%%%
\subsection{DFEM Formulation of the Diffusion Equation}
\label{sec::DFEM}
%%%%%%%%%%%%%%%%%%%%%%%%%%%%%

For this work we choose to employ an interior penalty formulation of the diffusion equation based of the work of Nitsche and Arnold \cite{ref::nitsche_IP,ref::arnold_1982_IP}. which has been called the Modified Interior Penalty Method (MIP) \cite{wang2009adaptive,ref::DSA_2D_arb_poly,ref::DSA_wang_ragusa}. The MIP weak form for the diffusion equation in Eq. (\ref{eq:DIffEq}) is given by:

\begin{equation}
a( \Phi, \Phi^*) = b(\Phi^*)
\label{eq:MIP_bilinear_form}
\end{equation}

\noindent with the following bilinear matrix:

\begin{equation}
\begin{aligned}
a( \Phi, \Phi^*) = \Big<  D \vec{\nabla}  \Phi , \vec{\nabla} \Phi^*  \Big>_{\mathcal{D}} + \Big<  \sigma_a   \Phi ,  \Phi^*  \Big>_{\mathcal{D}} \\
+  \frac{1}{2} \Big<    \Phi ,   \Phi^*\Big>_{\partial \mathcal{D}^r} +  \Big< \kappa_e^{MIP} [\![   \Phi ]\!] , [\![  \Phi^* ]\!]\Big>_{E_h^i}  \\ 
+ \Big<  [\![   \Phi ]\!] , \{\!\{  D \partial_n \Phi^* \}\!\}\Big>_{E_h^i} +\Big< \{\!\{  D \partial_n  \Phi \}\!\} , [\![  \Phi^* ]\!]\Big>_{E_h^i} \\
+ \Big< \kappa_e^{MIP}   \Phi ,   \Phi^* \Big>_{\partial \mathcal{D}^d} -\Big<   \Phi  ,  D \partial_n \Phi^* \Big>_{\partial \mathcal{D}^d} -\Big<   D \partial_n  \Phi ,   \Phi^*\Big>_{\partial \mathcal{D}^d}  
\end{aligned}
\label{eq::MIP_bilinear_op}
\end{equation}

\noindent and with the following right-hand-side form:

\begin{equation}
\begin{aligned}
b(\Phi^*) =& \Big<  Q, \Phi^*  \Big>_{\mathcal{D}}  - \Big<   J_{0}, \Phi^*  \Big>_{\partial \mathcal{D}^n} +  2 \Big<   J_{inc}, \Phi^*  \Big>_{\partial \mathcal{D}^r} 
\end{aligned}
\label{eq::MIP_rhs_op}
\end{equation}

\noindent where $\kappa_e^{MIP}$ is the MIP penalty coefficient and the mean value and the jump of the scalar flux on the interior faces, $E_h^i$, are given by 

\begin{equation}
\{\!\{ \Phi \}\!\} = \frac{\Phi^{+} + \Phi^{-}}{2} \qquad \text{and} \qquad  [\![ \Phi ]\!] = \Phi^{+} - \Phi^{-},
\label{eq::mean_and_jump}
\end{equation}

\noindent respectively.

The bilinear matrix is symmetric positive-definite and has been shown to be efficiently invertible with various preconditioning schemes \cite{ref::DSA_2D_arb_poly}.

%%%%%%%%%%%%%%%%%%%%%%%%%%%%%
\subsection{Basis Functions}
\label{sec::PWLD}
%%%%%%%%%%%%%%%%%%%%%%%%%%%%%

For this work, we wish to utilize FEM basis functions . For a given arbitrary polyhedral cell, $K$, the basis function at vertex $j$ is given by

\begin{equation}
b_j(x,y,z) = t_j(x,y,z) + \sum_{f=1}^{F_j} \beta_{f,j} t_f(x,y,z) + \alpha_{K,j} t_c(x,y,z), 
\label{eq::3D_PWLD_basis_functions}
\end{equation}

\noindent where $t_j$ is the standard linear function with unity at vertex $j$ that linearly decreases to zero to the cell center, face center, and each vertex that shares an edge with vertex $j$, $t_c$ is the cell "tent" function which is unity at the cell center and linearly decreases to zero to each cell vertex and face center, $\alpha_{K,j}$ is the weight parameter for vertex $j$ in cell $K$, $\beta_{f,j}$ is the weight parameter for face $f$ touching cell vertex $j$, and $F_j$ is the number of faces touching vertex $j$. Like the previous work defining the PWLD method \cite{bailey2008phd}, we also choose to assume the cell and face weighting parameters are


\begin{equation}
\alpha_{K,j} = \frac{1}{N_K} \qquad \text{and} \qquad \beta_{f,j} = \frac{1}{N_f},
\label{eq::PWLD_weight_vals}
\end{equation}

\noindent respectively, where $N_K$ is the number of vertices in cell $K$ and $N_f$ is the number of vertices on face $f$, which leads to constant values of $\alpha$ and $\beta$ for each cell and face, respectively. 

%%%%%%%%%%%%%%%%%%%%%%%%%%%%%%%%%%%%%%%%%%%%%%%%%%%%%%%%%%%%%%%%%%%%%%%%%%%%%%%%
\section{Results and Analysis}

To demonstrate 


%%%%%%%%%%%%%%%%%%%%%%%%%%%%%%%%%%%%%%%%%%%%%%%%%%%%%%%%%%%%%%%%%%%%%%%%%%%%%%%%
\section{Conclusions}

We have presented 

%%%%%%%%%%%%%%%%%%%%%%%%%%%%%%%%%%%%%%%%%%%%%%%%%%%%%%%%%%%%%%%%%%%%%%%%%%%%%%%%
%\appendix
%\section{Appendix}

%%%%%%%%%%%%%%%%%%%%%%%%%%%%%%%%%%%%%%%%%%%%%%%%%%%%%%%%%%%%%%%%%%%%%%%%%%%%%%%%
\section{Acknowledgments}
This material is based upon work supported by the Department of Energy Rickover Fellowship Program in Nuclear Engineering.

%%%%%%%%%%%%%%%%%%%%%%%%%%%%%%%%%%%%%%%%%%%%%%%%%%%%%%%%%%%%%%%%%%%%%%%%%%%%%%%%
\bibliographystyle{ans}
\bibliography{references}
\end{document}

